%%%%%%%%%%%%%%%%%%%%%%%%%%%%%%%%%%%%%%%%%%%%%%%%%%%%%%%%%
%  PATT2 FORM VERSION 02/2003 - COMPLETED EXAMPLE       %
%%%%%%%%%%%%%%%%%%%%%%%%%%%%%%%%%%%%%%%%%%%%%%%%%%%%%%%%%
%  ENTER THE INFORMATION BETWEEN THE CURLY BRACKETS     %
%%%%%%%%%%%%%%%%%%%%%%%%%%%%%%%%%%%%%%%%%%%%%%%%%%%%%%%%%
%  DO NOT EDIT THE PATT2.STY FILE IN ANY WAY            %
%%%%%%%%%%%%%%%%%%%%%%%%%%%%%%%%%%%%%%%%%%%%%%%%%%%%%%%%%

%%%%%%%%%%%%%%%%%%%%%%%%%%%%%%%%%%%%%%%%%%%%%%%%%%%%%%%%%
% the default font size must not be altered from 11pt.  %
% proposals written in a smaller font will be rejected  %
%%%%%%%%%%%%%%%%%%%%%%%%%%%%%%%%%%%%%%%%%%%%%%%%%%%%%%%%%

\documentclass[11pt]{article}
\usepackage{patt2}

% Allow epsfig, psfig or graphicx packages

\usepackage{epsfig}
\usepackage{psfig}
\usepackage{graphicx}

% Uncomment below if using pdflatex
%\usepackage{epstopdf}
%\pdfpageheight=11.69in
%\pdfpagewidth=8.26in

\typeout{This is the PATT2 (Optical/IR) example LaTeX form}

% add any personal macros here

% end macros

\begin{document}

\telescope{WHT}    % AAT, UKST, WHT, INT or UKIRT 
\semester {2003B}  % eg 2003B 

\category {5}   % Scientific Category (enter a number):
                % 1 Solar system and extrasolar planets
                % 2 ISM, CSM, PNe, including star formation
                % 3 Stars and stellar populations (galactic and circum-galactic)
                % 4 Low-z universe
                % 5 High-z universe
 
\relatedapps{x}{}{}{}{}{}{}{}{} % Coordinated PATT applications {x} 
% in order AAT,UKST,WHT,INT,UKIRT,JCMT,Gemini,LT,MERLIN


%%%%%%%%%%%%%%%%%%%%%%%%
% PAGE 1 OF PATT2 FORM %
%%%%%%%%%%%%%%%%%%%%%%%%

% PI information

\pisurname     {Wogan}             % Surname
\pititle       {Mr}                % Mr/Mrs/Ms/Miss/Dr/Prof.
\pifirstname   {Chris}             % First name
\pistatus      {Broadcaster}       % Post held
\piaddressone  {BBC}               % Name of Institute
\piaddresstwo  {Broadcasting House}% Postal Address
\piaddressthree{London W1A 4WW}    % Postal Address
\piphone       {0123-456789}       % Phone number
\pifax         {0987-654321}       % Fax number
\piemail       {tw@bbc.org}        % email address
\piobserver    {Yes}               % Is the PI going to observe? {Yes/No}

% collaborator 1

\collabonename    {B. Walters}     % Name of first collaborator
\collaboneinst    {CBS}            % Name of Institute
\collaboneobserver{No}             % Will collaborator observe? {Yes/No}

% collaborator 2

\collabtwoname    {}
\collabtwoinst    {}
\collabtwoobserver{}

% collaborator 3

\collabthreename    {}
\collabthreeinst    {}
\collabthreeobserver{}

% collaborator 4

\collabfourname    {}
\collabfourinst    {}
\collabfourobserver{}

% note additional collaborators can be added by inserting multiple
%names into these entries.



% proposal information 

\title      {To show that the Hubble Constant is in fact 42}% Brief title  (12 words only)

\abstract   {

% summary of proposed observations
% add your abstract here.  the font size must not be smaller 
% than the default in the style-file

It is suggested that the answer to the Universe and Everything is, to
within a jot or two, 42. By taking images with our 2-D colour gun we 
will verify this result.


}                


% Instrument requirements

\focalstation       {GHRIL}        % eg prime,f/3.3,f/8,f/15,f/36,cs/36
\instrument         {Wogan gun}    % eg RGO spec, UCLES, UHRF, Taurus, TTF, IRIS
                                   % prime focus imager, LDSS etc
\detector           {SITe}         % which chip do you want to use?
\gratingsandfilters {123R}         % eg UBVRI,H$\alpha$,1200R,etc

\timerequested  {6}{}{}{N}         % No. of {Dark},{Grey},{Bright}, 
                                   % {Weeks/Nights/Hours}
\minuseful      {6}{}{}            % Minimum number of useful {D},{G},{B} 
\lttotaltime    {}{}{}{}           % For long term proposals indicate the
                                   % TOTAL requested {D},{G},{B},{W/N/H}


%%%%%%%%%%%%%%%%%%%%%%%%
% PAGE 2 OF PATT2 FORM %
%%%%%%%%%%%%%%%%%%%%%%%%

\prefdates            {October}    % Preferred dates, eg Jan,Feb
\impossdates          {January}    % Impossible dates, eg Mar,Apr
\datesjustification   {RA's}       % Why impossible? eg wrong RAs, etc
\simultaneous         {With the AAT} % Simultaneous with other tels/satellites?
\othertimeconstraints {Must be dark} % eg Moon phase/position,specific dates
\serviceobservingyes  {}           % Observations to be done as Service? {x}
\serviceobservingno   {x}          % or not {x}
\serviceobservingmaybe{}           % or maybe {x}
\supporteverynight    {}           % Support astronomer every night? {x}
\supportnone          {}           % No support astronomer? {x}
\supportfirstnight    {x}          % Support astronomer first night only? {x}
                                   % (This is the only option for ING)

% target info                      % Target RA,Dec,Mags,Colours,Exp Time

\targetinfo{

%enter target information here.  this should include name, ra, dec
%and some indication of magnitude, line flux etc.  it is OK to use
%a small font here.

\begin{tabular}[t]{p{1.2in}p{1.0in}p{1.0in}p{1.0in}p{1.0in}p{1.0in}}
  Anon    & 12 23 & +44 30 & 16.5 & 0.4 & 2 hrs \\
  Anon2   & 15 23 & +24 30 & 12.3 & 0.2 & 2 mins \\
  Anon3   & 17 23 & +44 30 & 16.5 & 0.4 & 2 hrs \\
  Anon4   & 19 23 & +24 30 & 12.3 & 0.2 & 2 mins \\
  Anon5   & 21 23 & +44 30 & 16.5 & 0.4 & 2 hrs \\
  Anon6   & 23 23 & +24 30 & 12.3 & 0.2 & 2 mins \\
  Anon7   & 01 23 & +44 30 & 16.5 & 0.4 & 2 hrs \\
  Anon8   & 03 23 & +24 30 & 12.3 & 0.2 & 2 mins \\
  Anon9   & 05 23 & +44 30 & 16.5 & 0.4 & 2 hrs \\
\end{tabular}
}

% LIST ALL SIMILAR/SUPPORTING APPLICATIONS TO ANY PATT OR OTHER TIME
% ASSIGNMENT COMMITTEE  
% You must include a brief description of any
% other applications whose targets or science goals are similar to 
% those requested here

\otherapplications{
\begin{tabular}[t]{p{1.6in}p{5.0in}}
%Telescope/Committee & Short title of programme  \\
AAT & The Hubble constant in the Southern Hemisphere \\
VLT & The Hubble constant from interferometry \\
\end{tabular}
}


%%%%%%%%%%%%%%%%%%%%%%%%%%%%%%%%%%%%%%%%%%%%%%%%%%%
% PAGE 3 OF PATT2 FORM - SCIENTIFIC JUSTIFICATION %
%%%%%%%%%%%%%%%%%%%%%%%%%%%%%%%%%%%%%%%%%%%%%%%%%%%

\sciencecase{

There is very little that we need to do to justify this application
as it has in fact been shown ({\it See Hitch Hikers Guide to the Galaxy})
that the answer to Life, The Universe and Everything is without doubt, 42. It
might be worth while, however, to add a little observational confirmation to
this if only to while away the long winter nights....

% add your science case here.  

% proposals written in a font smaller than 11pt will be rejected

}

%%%%%%%%%%%%%%%%%%%%%%%%%%%%%%%%%%%%%%%%%%%%%%%%%%%%%%%%
% PAGE 3a OF PATT2 FORM - SCIENTIFIC JUSTIFICATION     %
% FOR PROPOSALS TO AAT, WHT or UKIRT FOR 8 OR MORE     % 
% NIGHTS, AND FOR  ALL (I.E. INCLUDING INT AND UKST)   %
% LONG-TERM AND COORDINATED PROPOSALS (INCLUDING THOSE %
% COORDINATED WITH NON-PATT TELESCOPES)                %
%%%%%%%%%%%%%%%%%%%%%%%%%%%%%%%%%%%%%%%%%%%%%%%%%%%%%%%%

\extendedsciencecase{

% continue your science case here ONLY if applying for 8 or 
% more nights to the AAT, WHT OR UKIRT, or if your proposal 
% is a long-term (multi-semester) proposal to the AAT, UKST, 
% WHT, INT or UKIRT, or if your proposal (to AAT, UKST, WHT, 
% INT or UKIRT) is coordinated with other telescopes (including 
% non-PATT telescopes).  

% Remember to comment IN the \makepatttwopagethreea command later 
% in this file if you have written an extended science case  

% proposals written in a font smaller than 11pt will be rejected

...especially 8 or more nights on the AAT, WHT and UKIRT.

}


%%%%%%%%%%%%%%%%%%%%%%%%%%%%%%%%%%%%%%%%%%%%%%%%%%%%%%%%%%%%%%%%%%%%%%%%%%%%
% PAGE 4 OF PATT2 FORM - TECHNICAL INFORMATION (I) - FEASIBILITY, S/N, ETC %
%%%%%%%%%%%%%%%%%%%%%%%%%%%%%%%%%%%%%%%%%%%%%%%%%%%%%%%%%%%%%%%%%%%%%%%%%%%%

\technicalpage{

no problems!

% add any technical details you wish to transmit to the panel and
% technical assessor here. you may also include references here
% as well as on the next page

% proposals written in a font smaller than 11pt will be rejected


}

%%%%%%%%%%%%%%%%%%%%%%%%%%%%%%%%%%%%%%%%%%%%%%%%%%%%%%%%%%%%%%%%%%%%%%%%%%%%%%
% PAGE 4a OF PATT2 FORM - TECHNICAL INFORMATION (II) - REFERENCES, FIGS, ETC %
%%%%%%%%%%%%%%%%%%%%%%%%%%%%%%%%%%%%%%%%%%%%%%%%%%%%%%%%%%%%%%%%%%%%%%%%%%%%%%

\figsandrefspage{

Complete Works of Shakespeare, W.Shakespeare.\newline
\hfill Hitchiker's Guide to the Galaxy, D.Adams.\newline
\hfill Anon,B. MNRAS  1099, 567, 1999.\newline

% Here is an example of how to embed a PostScript figure in LaTeX using psfig
\psfig{file=patt2_examplefig.ps,angle=270,width=15cm}

% Here is an example of how to embed a PostScript figure in LaTeX using graphicx
%\includegraphics[width=15cm,angle=270]{patt2_examplefig.ps}

% add any figures, references etc here.  remember to embed your figures.

% proposals written in a font smaller than 11pt will be rejected

}

%%%%%%%%%%%%%%%%%%%%%%%%
% PAGE 5 OF PATT2 FORM %
%%%%%%%%%%%%%%%%%%%%%%%%

\backupprogram{Observe brighter members of the sample...} % Summary of backup programme

\previous{            % Previous applications (last 4 Sems)
\begin{tabular}[t]{p{1.5in}p{0.7in}p{0.8in}p{3.2in}}
                      %Patt No. & Award & clear nights & comments \\
W/2003A/99 & 4Gn & 4 & sub-arcsecond seeing throughout  \\
\end{tabular}
}

\publications     {none} % List pubs with data from patt time (last 4 Sems)

\experience       {none} % Experience of observers on other telescopes
\graduatestudent  {Fran Godfrey}{Hubble}       % Research student {Name of student}{Project}
\grant            {}{}{} % {Name of PI}{Grant title}{Grant No.}
\nonstandardtravel{Wogan travels first class}         % Justify T&S for more than one UK observer
\otherexpenditure {none} % eg for freight etc.


%%%%%%%%%%%%%%%%%%%%%%%%%%%%%%%%%%%%%%%%%%%%%%%%%%%%%%%%%%%%%%%%%%%%%%%%%%%%%
% PAGE 6 OF PATT2 FORM - THIS SECTION IS REMOVED BY THE AAT DURING PROCESSING
% IF INCLUDED.  ING APPLICANTS SHOULD NOT ENTER ANYTHING HERE.
%%%%%%%%%%%%%%%%%%%%%%%%%%%%%%%%%%%%%%%%%%%%%%%%%%%%%%%%%%%%%%%%%%%%%%%%%%%%%

\shorttitle {}           % Ignore if you are an ING applicant.



\makepatttwopageone
\makepatttwopagetwo
\makepatttwopagethree

%%%%%%%%%%%%%%%%%%%%%%%%%%%%%%%%%%%%%%%%%%%%%%%%%%%%%%%%%%%%%%%%%%%%%
% COMMENT THE FOLLOWING LINE IN *ONLY* IF                           % 
%                                                                   %
% YOU ARE APPLYING FOR 8 OR MORE NIGHTS ON THE AAT, WHT OR UKIRT, OR% 
% YOUR PROPOSAL IS LONG-TERM FOR THE AAT, WHT, UKIRT OR INT, OR     % 
% YOUR PROPOSAL IS COORDINATED WITH OTHER TELESCOPES                % 
%                                                                   % 
% *AND* YOU HAVE USED THE CONTINUATION PAGE FOR YOUR SCIENTIFIC CASE% 
%%%%%%%%%%%%%%%%%%%%%%%%%%%%%%%%%%%%%%%%%%%%%%%%%%%%%%%%%%%%%%%%%%%%%
 
\makepatttwopagethreea

\makepatttwopagefour
\makepatttwopagefoura
\makepatttwopagefive

% FOR ING APPLICATIONS LEAVE THE FOLLOWING LINE COMMENTED OUT

%\makepatttwopagesix         

% FOR AAT APPLICATIONS YOU MAY LEAVE IT IN IF YOU WISH.  IT WILL NOT
% BE TRANSMITTED TO THE TAG HOWEVER.

\end{document}
